
\section{Statistical mechanics}

The basic idea of LBE is to replace Boolean occupation numbers $n_i$ in LGCA with the corresponding ensemble-averaged populations. It can be shown, that Boltzmann equation

\begin{equation}
\frac{\partial f}{\partial t} + \overrightarrow{v} \nabla f = Q
\end{equation}

can be referred to LGCA discrete kinetic equation

\begin{equation}
n_i(\overrightarrow{x} + \overrightarrow{c}_i \Delta t, t + \Delta t) = n_i(\overrightarrow{x},t) +\Delta_i
\end{equation}

by expansion of the left hand side

\begin{equation}
n_i(\overrightarrow{x} + \overrightarrow{c}_i \Delta t, t + \Delta t) = n_i(\overrightarrow{x},t) + \Delta t \frac{\partial n_i}{\partial t} + \overrightarrow{c}_i \Delta{t} \nabla{n_i} + O\big(\Delta t)^2\big)
\end{equation}
and if we will substitute $n_i \rightarrow f$, $\overrightarrow{c}_i \rightarrow \overrightarrow{v}$, $\frac{\Delta_i}{\Delta t} \rightarrow Q$, then we will get Boltzmann equation. Q collision integral is very complex and is very complicated to compute, that is why it was proposed to replace it with BGK (Bhatnagar-Gross-Krook) approximation

\begin{equation}
\frac{\partial f}{\partial t} + \overrightarrow{v} \nabla f = -\frac{1}{\tau}(f - f^{(eq)})
\end{equation}
Where $f^{eq}$ is Maxwell Boltzmann equilibrium distribution. The idea behind BGK is that our model should follow Maxwell–Boltzmann distribution law, and the difference inside of collision operator gives us the deviation of computed $f$ from  Maxwell Boltzmann equilibrium distribution.

At the end we discretize Boltzmann equations to get Lattice-Boltzmann method

\begin{equation}
f_i(\overrightarrow{x} + \overrightarrow{c}_i \Delta t, t + \Delta t) = f_i(\overrightarrow{x},t) + -\frac{1}{\tau} \big(f_i(\overrightarrow{x},t) - f_i^{(eq)}(\rho(\overrightarrow{x},t), u(\overrightarrow{x},t))\big)
\end{equation}
where $\rho$ and $u$ are macroscopic density and velocity of the flow.

For the lattice BGK model, the equilibrium distribution function can be calculated as

\begin{equation}
f_i^{(eq)} = f_i^{(eq)}(\rho, u) = t_p \rho ( 1 + \frac{3}{c^2} c_i u + \frac{9}{2c^4} (c_i u)^2 - \frac{3u^2}{2c^2})
\end{equation}
where $c_i$ is the corresponding lattice velocity, $c$ is the velocity of sound and $t_p$ is a direction dependent parameter, which depends on DdQq model. Where d is the number of dimensions and q is the number of directions in which velocities are discretized.

In case of the D2Q9 model, $t_p$ can be derived as

\begin{equation}
t_p = \frac{4}{9} \quad \textrm{for} \quad ||c_i||=0
\end{equation}
\begin{equation}
t_p = \frac{1}{9} \quad \textrm{for} \quad ||c_i||=1
\end{equation}
\begin{equation}
t_p = \frac{1}{36} \quad \textrm{for} \quad ||c_i||=\sqrt{2}
\end{equation}
and in case of the D3Q19 model as

\begin{equation}
t_p = \frac{1}{3} \quad \textrm{for} \quad ||c_i||=0
\end{equation}
\begin{equation}
t_p = \frac{1}{18} \quad \textrm{for} \quad ||c_i||=1
\end{equation}
\begin{equation}
t_p = \frac{1}{36} \quad \textrm{for} \quad ||c_i||=\sqrt{2}
\end{equation}

For both the D2Q9 and the D3Q19 model, the parameter $c$ can be considered to be 1.