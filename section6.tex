
\section{Simple boundary conditions}

We cannot build any fluid flows model without including of the influence of the surrounding environment. This influence can be described mathematically with so named boundary conditions. Boundary conditions build the external constraints which influence the solution of whole system. The implementation of Boundary conditions in LBM are straight forward, but the mathematical background of that Boundary conditions may be even more complicated, then the LBM itself.

One of the easiest boundary conditions is no-slip. The no-slip boundary conditions are used in case of real walls or obstacles, which offer a certain amount of friction to the fluid. In the case, when the fluid velocity at the wall is reduced to zero, the Boundary conditions can be implemented in the following way: all distribution functions of a fluid cell neighboring an obstacle cell pointing towards the obstacle are simply reversed [1]. This assures, that the fluid velocity normal and tangential to the wall is zero.

\begin{figure}[H]
  \centering
  \begin{subfigure}[h]{0.3\textwidth}
    \includegraphics[width=\textwidth]{img/fig8-1.png}
  \end{subfigure}
  \begin{subfigure}[h]{0.3\textwidth}
    \includegraphics[width=\textwidth]{img/fig8-2.png}
  \end{subfigure}
  \caption{No-slip Boundary conditions. Left - before streaming step; Right – after streaming step.}
\end{figure}

The next simple Boundary conditions are free-slip Boundary conditions. They can be used if we have walls without friction. An usual example of free-slip Boundary conditions is at symmetry axes in certain simulation domains, for example in the middle of a channel flow. The free-slip Boundary conditions are applied as follows: all distribution functions of a fluid cell neighboring a free-slip boundary pointing towards the boundary are reflected in their component normal to the wall. This assures, that the velocity normal to the wall is zero, whereas the velocity tangential to the wall remains unchanged [1]:

\begin{equation}
\frac{\partial v_t}{\partial n} = 0.
\end{equation}

\begin{figure}[H]
  \centering
  \begin{subfigure}[h]{0.3\textwidth}
    \includegraphics[width=\textwidth]{img/fig9-1.png}
  \end{subfigure}
  \begin{subfigure}[h]{0.3\textwidth}
    \includegraphics[width=\textwidth]{img/fig9-2.png}
  \end{subfigure}
  \caption{Free-slip Boundary conditions. Left - before streaming step; Right – after streaming step.}
\end{figure}

To implement the lid-driven cavity, which I will describe in the next section, we need one more Boundary conditions. Such Boundary conditions is called moved no-slip boundaries. With such Boundaries we have some certain velocity of the wall, which is imparted to the fluid particles and accelerate the fluid. These boundary is similar to no-slip boundary except of the scaling term depending on the wall velocity [1]:

\begin{equation}
f_{\bar{\alpha}}(\vec{x},t) = f_{\alpha}(\vec{x},t) - 2 t_p \rho \frac{3}{c^2} c_{\alpha} u_{w}
\end{equation}
where $\alpha$ is the direction towards the wall, $\bar{\alpha}$ is the reverse direction from the wall, $t_p$ is the direction dependent parameter, $\rho$ is the fluid density of the near wall fluid cell, and $u_w$ is the wall velocity.
